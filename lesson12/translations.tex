\documentclass[12pt]{article}
\usepackage[T1]{fontenc}
\usepackage[utf8]{inputenc}
\usepackage[margin=1in]{geometry}
\usepackage{fancyhdr}
\usepackage[sc]{titlesec}
\usepackage{enumerate}

\pagestyle{fancy}
\fancyhead{}
\fancyfoot{}
\fancyhead[L]{\slshape \MakeUppercase{Uses of the Dative}}
\fancyhead[R]{\slshape \MakeUppercase{Lesson XII}}
\fancyfoot[C]{\thepage}
\renewcommand{\footrulewidth}{0pt}

\begin{document}

\section{\textsc{Exercises}}
\begin{enumerate}[I.]
	\setlength{\itemsep}{1em}
	\item \begin{enumerate}[1)]
		\item He is, he was, he will be.
		\item Are they? they were, they will be.
		\item We are, we were, we shall be.
		\item I am, I was, I shall be.
		\item Are you? you were, you will be.
		\item You are, you were, you will be.
		\item To be.
		\item The master has a book.
		\item The master has a book.
		\item Duilius has many boys.
		\item Duilius has many boys.
		\item Were they for the long and rough road?
		\item The messengers will be wretched.
		\item The black sky is not pleasing to the sailors.
	\end{enumerate}
	\item \begin{enumerate}[1)]
		\item Sum, sumus.
		\item Erāmusne? eram.
		\item Erō, erimus.
		\item Estne? sunt.
		\item Erat, erant.
		\item Erit, erunt.
		\item Es, erās.
		\item Eris, esse.
		\item Vir līber eris.
		\item Galbae est equus niger.
		\item Virō erant fīliī pigrī et miserī.
		\item Dōna Mārcō amīcō meō erunt grāta.
		\item Līberī bonī līberōs pigrōs superant.
	\end{enumerate}
\end{enumerate}

\section{\textsc{Review}}
\begin{enumerate}[1.]
	\item The declensions of \textbf{servus}, \textbf{vir}, and \textbf{magister}.
	\begin{center}
		\begin{tabular}{ c c c c c c }
		& & & \textsc{Singular} & & \\
		& & & & & \\
		\textit{Nom}. & servus & & vir & & magister \\
		\textit{Gen}. & servī & & virī & & magistrī \\
		\textit{Dat}. & servō & & virō & & magistrō \\
		\textit{Acc}. & servum & & virum & & magistrum \\
		\textit{Abl}. & servō & & virō & & magistrō \\
		& & & & & \\
		& & & \textsc{Plural} & & \\
		& & & & & \\
		\textit{Nom}. & servī & & virī & & magistrī \\
		\textit{Gen}. & servōrum & & virōrum & & magistrōrum \\
		\textit{Dat}. & servīs & & virīs & & magistrīs \\
		\textit{Acc}. & servōs & & virōs & & magistrōs \\
		\textit{Abl}. & servīs & & virīs & & magistrīs \\
		\end{tabular}
	\end{center}
	\item The declensions of \textbf{cārus}, \textbf{niger}, and \textbf{tener} in the three genders.
	\begin{center}
		\begin{tabular}{ c c c c c c }
		& & & \textsc{Singular} & & \\
		& & & & & \\
		& \textsc{Masc}. & & \textsc{Fem}. & & \textsc{Neut}. \\
		\textit{Nom}. & cārus & & cāra & & cārum \\
		\textit{Gen}. & cārī & & cārae & & cārī \\
		\textit{Dat}. & cārō & & cārae & & cārō \\
		\textit{Acc}. & cārum & & cāram & & cārum \\
		\textit{Abl}. & cārō & & cārā & & cārō \\
		& & & & & \\
		& & & \textsc{Plural} & & \\
		& & & & & \\
		& \textsc{Masc}. & & \textsc{Fem}. & & \textsc{Neut}. \\
		\textit{Nom}. & cārī & & cārae & & cāra \\
		\textit{Gen}. & cārōrum & & cārārum & & cārōrum \\
		\textit{Dat}. & cārīs & & cārīs & & cārīs \\
		\textit{Acc}. & cārōs & & cārās & & cāra \\
		\textit{Abl}. & cārīs & & cārīs & & cārīs \\
		\end{tabular}
	\end{center}
	\begin{center}
		\begin{tabular}{ c c c c c c }
		& & & \textsc{Singular} & & \\
		& & & & & \\
		& \textsc{Masc}. & & \textsc{Fem}. & & \textsc{Neut}. \\
		\textit{Nom}. & niger & & nigra & & nigrum \\
		\textit{Gen}. & nigrī & & nigrae & & nigrī \\
		\textit{Dat}. & nigrō & & nigrae & & nigrō \\
		\textit{Acc}. & nigrum & & nigram & & nigrum \\
		\textit{Abl}. & nigrō & & nigrā & & nigrō \\
		& & & & & \\
		& & & \textsc{Plural} & & \\
		& & & & & \\
		& \textsc{Masc}. & & \textsc{Fem}. & & \textsc{Neut}. \\
		\textit{Nom}. & nigrī & & nigrae & & nigra \\
		\textit{Gen}. & nigrōrum & & nigrārum & & nigrōrum \\
		\textit{Dat}. & nigrīs & & nigrīs & & nigrīs \\
		\textit{Acc}. & nigrōs & & nigrās & & nigra \\
		\textit{Abl}. & nigrīs & & nigrīs & & nigrīs \\
		\end{tabular}
	\end{center}
	\begin{center}
		\begin{tabular}{ c c c c c c }
		& & & \textsc{Singular} & & \\
		& & & & & \\
		& \textsc{Masc}. & & \textsc{Fem}. & & \textsc{Neut}. \\
		\textit{Nom}. & tener & & tenera & & tenerum \\
		\textit{Gen}. & tenerī & & tenerae & & tenerī \\
		\textit{Dat}. & tenerō & & tenerae & & tenerō \\
		\textit{Acc}. & tenerum & & teneram & & tenerum \\
		\textit{Abl}. & tenerō & & tenerā & & tenerō \\
		& & & & & \\
		& & & \textsc{Plural} & & \\
		& & & & & \\
		& \textsc{Masc}. & & \textsc{Fem}. & & \textsc{Neut}. \\
		\textit{Nom}. & tenerī & & tenerae & & tenera \\
		\textit{Gen}. & tenerōrum & & tenerārum & & tenerōrum \\
		\textit{Dat}. & tenerīs & & tenerīs & & tenerīs \\
		\textit{Acc}. & tenerōs & & tenerās & & tenera \\
		\textit{Abl}. & tenerīs & & tenerīs & & tenerīs \\
		\end{tabular}
	\end{center}
	\item The declensions of \textbf{oppidānus līber}, \textbf{nauta asper}, \textbf{dōnum pulchrum}, and \textbf{poēta clārus}.
	\begin{center}
		\begin{tabular}{ c c c c }
		& & \textsc{Singular} & \\
		& & & \\
		\textit{Nom}. & oppidānus līber & & nauta asper \\
		\textit{Gen}. & oppidānī līberī & & nautae asperī \\
		\textit{Dat}. & oppidānō līberō & & nautae asperō \\
		\textit{Acc}. & oppidānum līberum & & nautam asperum \\
		\textit{Abl}. & oppidānō līberō & & nautā asperō \\
		& & & \\
		& & \textsc{Plural} & \\
		& & & \\
		\textit{Nom}. & oppidānī līberī & & nautae asperī \\
		\textit{Gen}. & oppidānōrum līberōrum & & nautārum asperōrum \\
		\textit{Dat}. & oppidānīs līberīs & & nautīs asperīs \\
		\textit{Acc}. & oppidānōs līberōs & & nautās asperōs \\
		\textit{Abl}. & oppidānīs līberīs & & nautīs asperīs \\
		\end{tabular}
	\end{center}
	\begin{center}
		\begin{tabular}{ c c c c }
		& & \textsc{Singular} & \\
		& & & \\
		\textit{Nom}. & dōnum pulchrum & & poēta clārus \\
		\textit{Gen}. & dōnī pulchrī & & poētae clārī \\
		\textit{Dat}. & dōnō pulchrō & & poētae clārō \\
		\textit{Acc}. & dōnum pulchrum & & poētam clārum \\
		\textit{Abl}. & dōnō pulchrō & & poētā clārō \\
		& & & \\
		& & \textsc{Plural} & \\
		& & & \\
		\textit{Nom}. & dōna pulchra & & poētae clārī \\
		\textit{Gen}. & dōnōrum pulchrōrum & & poētārum clārōrum \\
		\textit{Dat}. & dōnīs pulchrīs & & poētīs clārīs \\
		\textit{Acc}. & dōna pulchra & & poētās clārōs \\
		\textit{Abl}. & dōnīs pulchrīs & & poētīs clārīs \\
		\end{tabular}
	\end{center}
	\item The following adjectives in \textbf{-er} that have occured keep the \textbf{e} in declension: asper, līber, miser, tener.
	\item Librōs; \textbf{Rule}. --- \textit{The direct object of a transitive verb is in the accusative}. Puerō; \textbf{Rule}. --- \textit{The indirect object is in the dative}. Cāra et grāta; \textbf{Rule}. --- \textit{Is a noun or an adjective used after certain intransitive or passive verbs to complete their meaning, and to describe or define the subject}. Amīcus; \textbf{Rule}. --- \textit{An appositive agrees in case with the noun which it limits}. Puellīs; \textbf{Rule}. --- \textit{The dative is used with \textbf{est, sunt}, etc., to denote the possessor, the thing possessed being the subject}. Duīlī; \textbf{Rule}. --- \textit{A noun used to limit another, and not denoting the same person or thing, is in the genitive}. Rosae tenerae; \textbf{Rule}. --- \textit{Adjectives meaning \textbf{near}, also \textbf{fit, friendly, pleasing, like} and their opposites, take the dative}.
\end{enumerate}

\section{\textsc{Conversation}}
\begin{enumerate}[1.]
	\item Whom is Sextus, your friend, teaching? He is teaching the son of Duilius.
	\item Was the trumpet pleasing to the boy? The trumpet was dear and pleasing to the boy.
	\item Who has the tender roses? The boys have the tender roses.
	\item Why do you show the pretty books to the master? Because he loves the books.
	\item O (my) messenger, do you see the new moon? I see the moon in the black sky.
\end{enumerate}

\end{document}
