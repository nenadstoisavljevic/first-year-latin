\documentclass[12pt]{article}
\usepackage[T1]{fontenc}
\usepackage[utf8]{inputenc}
\usepackage[margin=1in]{geometry}
\usepackage{fancyhdr}
\usepackage[sc]{titlesec}
\usepackage{enumerate}

\pagestyle{fancy}
\fancyhead{}
\fancyfoot{}
\fancyhead[L]{\slshape \MakeUppercase{Second Declension}}
\fancyhead[R]{\slshape \MakeUppercase{Lesson X}}
\fancyfoot[C]{\thepage}
\renewcommand{\footrulewidth}{0pt}

\begin{document}

\section{\textsc{Review}}
\begin{enumerate}[1.]
	\item The vocative singular of nouns in \textbf{-us} of the second declension has a special form in \textbf{ě}.
	\item The dative and ablative singular cases of the second declension end in \textbf{ō}. The dative and ablative plural cases of the second declension end in \textbf{īs}.
	\item The ending of the accusative singular is in \textbf{um}.
	\item The nominative and accusative plural cases of the neuter end in \textbf{a}.
	\item The declension of \textbf{dominus}, \textit{master}.
	\begin{center}
		\begin{tabular}{ c c c }
		& \textsc{Singular} & \textsc{Plural} \\
		& & \\
		\textit{Nom}. & dominus & dominī \\
		\textit{Gen}. & dominī & dominōrum \\
		\textit{Dat}. & dominō & dominīs \\
		\textit{Acc}. & dominum & dominōs \\
		\textit{Abl}. & dominō & dominīs
		\end{tabular}
	\end{center}
	\item The declension of \textbf{bellum}, \textit{war}.
	\begin{center}
		\begin{tabular}{ c c c }
		& \textsc{Singular} & \textsc{Plural} \\
		& & \\
		\textit{Nom}. & bellum & bella \\
		\textit{Gen}. & bellī & bellōrum \\
		\textit{Dat}. & bellō & bellīs \\
		\textit{Acc}. & bellum & bella \\
		\textit{Abl}. & bellō & bellīs
		\end{tabular}
	\end{center}
	\item The declension of \textbf{nūntius}, \textit{messenger}.
		\begin{center}
		\begin{tabular}{ c c c }
		& \textsc{Singular} & \textsc{Plural} \\
		& & \\
		\textit{Nom}. & nūntius & nūntiī \\
		\textit{Gen}. & nūntī & nūntiōrum \\
		\textit{Dat}. & nūntiō & nūntiīs \\
		\textit{Acc}. & nūntium & nūntiōs \\
		\textit{Abl}. & nūntiō & nūntiīs
		\end{tabular}
	\end{center}
\end{enumerate}

\section{\textsc{Exercises}}
\begin{enumerate}[I.]
	\setlength{\itemsep}{1em}
	\item \begin{enumerate}[1)]
		\item Sons of the free men.
		\item The oar of the fierce sailor.
		\item For the fierce winds.
		\item In the wretched wagon.
		\item Food of the poor boys.
		\item O (my) dear son!
		\item Does he call? Do they overcome?
		\item We call the good boys.
		\item Many children are on the wide road.
		\item The poor messenger tells a wretched story.
		\item Many men are in the free town.
		\item What do the tender girls have?
	\end{enumerate}
	\item \begin{enumerate}[1)]
		\item Dominī servōrum miserōrum.
		\item Bellīs miserīs.
		\item Tuīs fīliīs parvīs.
		\item In hortīs nūntiōrum.
		\item Ō miser amīce!
		\item Fīlius Duīlī est līber.
		\item Pīrātās asperōs vidēmus.
		\item Puerōs parvōs et fēminās tenerās terrent.
		\item Incolae inimīcōs superant.
		\item Virī līberī sumus.
		\item In terrā līberōrum habitāmus.
		\item Fīliī virōrum līberōrum sumus.
	\end{enumerate}
\end{enumerate}

\section{\textsc{Conversation}}
\begin{enumerate}[1.]
	\item Who is showing the man a way? The kind goddess is showing the way.
	\item What is shining in the clear sky? The new moon is shining.
	\item Strong men, whom do you overcome? We overcome the fierce enemies.
	\item Do the boys have javelins? They have the long javelins of Duilius.
	\item To whom does the messenger tell a story? He tells the free men a story.
\end{enumerate}

\end{document}
