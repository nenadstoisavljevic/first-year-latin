\documentclass[12pt]{article}
\usepackage[T1]{fontenc}
\usepackage[utf8]{inputenc}
\usepackage[margin=1in]{geometry}
\usepackage{fancyhdr}
\usepackage[sc]{titlesec}
\usepackage{enumerate}

\pagestyle{fancy}
\fancyhead{}
\fancyfoot{}
\fancyhead[L]{\slshape \MakeUppercase{Second Declension}}
\fancyhead[R]{\slshape \MakeUppercase{Lesson XI}}
\fancyfoot[C]{\thepage}
\renewcommand{\footrulewidth}{0pt}

\begin{document}

\section{\textsc{Exercises}}
\begin{enumerate}[I.]
	\setlength{\itemsep}{1em}
	\item \begin{enumerate}[1)]
		\item Men, the inhabitants of the town.
		\item On the beautiful fields of the farmer.
		\item The beautiful books of the teachers.
		\item Of the fierce winds.
		\item Of the slow servants.
		\item For the sailor, my friend.
		\item White and black horses.
		\item For the wretched boys of Duilius.
		\item I see horses on the field of Marcus, (my) friend.
		\item The servant loves Sextus, (my) friend.
		\item The book, a beautiful gift of Cornelia, delights the master.
		\item Is the sky beautiful?
		\item A good servant is not slow.
		\item Many farmers have black horses.
		\item The beautiful girls are daughters of the mistress.
		\item Your books are a pleasing gift.
	\end{enumerate}
	\item \begin{enumerate}[1)]
		\item Galbae amīcō meō librum dōnō.
		\item Līberīs Sextī dona multa dōnāmus.
		\item Fīliīs virōrum librōs pulchrōs donant.
		\item Virī miserō in oppidō habitant.
		\item In Graeciā terrā asperā Eurōpae habitant.
		\item Multī sunt incolae in oppidīs Britanniae.
	\end{enumerate}
\end{enumerate}

\end{document}
